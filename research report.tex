\documentclass[12pt]{article}
\begin{document}
	\title{DRAMA IN THE NEVER ENDING
		PROBLEMS AT THE HILL OF THE ELITES.}
	\author{Kikomeko musa Reg no.:15/u/6675/ps Student no.:215004259}
	\maketitle
	
	\section{THINGS GO WRONG}
	It was a Monday afternoon, the sun was up in the sky and the day
	seemed brighter than yesterday the environment was very calm and
	silence had befallen all around the place, the population was very
	scarce with very few students moving around and several number of
	police cars patrolling along the main roads.
	It was just moments ago when the Makerere university main campus
	had held its mega strike that was termed as the Black Bloody Monday
	as formalized by the student leaders. There was a lot of tension
	amongst both the students and the security authorities for what was
	to happen next after the short period of relaxation in violence that
	had happened earlier. Surprisingly, the situation remained calm until
	the end of the day something that would never be expected to happen.
	\section{MIRACLES AS A SURPRISE}
	Unfortunately, the calmness in the place was short lived as the tension
	and violence was resumed in and outside campus when what is termed
	as the executive order was issued by his excellence the president of the
	republic of Uganda Mr.Yoweri Kaguta Museveni to close Makerere
	university with immediate effect.
	To him, he says that this directive issuance powers were given to him
	by the constitution of the republic of Uganda. The basis of issuing the
	directive was that he aimed at protecting people's lives and property.
	This executive order was issued at around 7:00 pm and since then
	tension increased every second that by passed.
	During that period of increased tension amongst all parties concerned including students, lecturers, university authorities, the people around the university and the whole world at large. The situation was intensified by the claims by the student leaders to mobilize students for a strike against the presidents directive.
	These threats of mobilizing another strike by the students t4triggered
	anger to the security authorities who claimed that by mobilizing a
	strike, students left them without any other option but to apply force.
	The students were then caught un aware when they found themselves
	being woken up by police coffers at dusk (wee hours) to pack their
	property and get out of the university.
	The police claimed that this was a preventive move to minimize any
	form of violence that was expected to happen when the day breaks.
	All students residing with the university hall were ordered to vacate
	the university premises by 8:00 am and those students that resided
	in university affiliates like hostels and rentals were given two days to
	also vacate them and go back home.
	The student guild went into dialogue with the university, the university management went into dialogue with the the president.
	The issue was ignored and nearly forgotten with time and would
	only be remembered and re-ignited by news headlines where they
	identified different parties going into discussion and dialogue over the
	closure of makerere university, this continued for several weeks and
	students regained panic when several dates started to be identified
	over which the ivory tower could be opened.
	A committee was also established to investigate about the issues of
	makerere and the continuous strikes that always tend to happen at
	the hill of elites, oldest and most prestigious university in Uganda.
	
	\section{NEVER SAY NEVER}
	To the surprising bit of the story, it was one day around the afternoon
	when the lecturers broke the silence and stopped their industrial
	strike that had lasted for approximately three months. This was
	indeed a surprise because it was not expected and neither had it
	been predicted earlier.
	The putting of the industrial strike at a stop over meant that it was
	away forward for the opening of the ivory tower and within a week
	initiatives to see that makerere university opens were on going until
	an official date for its opening was identified.
	The controversial strike that led to the uncalled for closure of the university by the presidents directive and the
	dramatic ending of the industrial strike and opening of the university
	leaves a lot of un answered questions to the civilians.
	
		\section{FINALLY RESOLVED}
	Following the establishment of the visitation committee to investigate the issues of the Ivory towel a lot happened and it's recommendations that were thought to be hopeful to solve the prevailing issues nearly worsened the situation and raised a lot of un answered questions.
	But after all those dramatic and chaotic events the solution to Makerere problems was identified because since it's closure students have never indulged them solves in any other strike.
		\section{.}
		THIS IS CATEGORIZED AS DESCRIPTIVE TYPE OF RESEARCH
\end{document}

